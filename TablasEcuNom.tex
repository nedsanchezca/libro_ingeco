 %-----------------------------------------------------------
 %              TABLA DE NOMENCLATURA
 %------------------------------------------------------
\graphicspath{ {img/tablas}, {img/} }
\begin{center}
\section*{i. Tabla de nomenclatura }
\addcontentsline{toc}{section}{i. Tabla de nomenclaturas }
\end{center}

\begin{spacing}{1}
\begin{center}
\begin{tabular}{ |p{2.5cm}|p{9.5cm}|}
\hline 
\rowcolor{orange!50}
\begin{center} \textbf{Siglas} \end{center} & \begin{center}

\textbf{Nombre}\end{center} \\ \hline     

EA & Efectivo Anual (tasa de referencia SuperFinanciera-nominal anual año vencido- naav)  \\ \hline

g & Gradiente geométrico   \\ \hline

H & Valor incremento escalón monetario  \\ \hline

I & Monto de interés  \\ \hline

IPC & Índice de precios al consumidor \\ \hline

L &  Gradiente aritmético monetario \\ \hline

$i$ & Tasa de interés periódico  \\ \hline

$j$ & Tasa de interés nominal anual (período) vencido  \\  \hline 

i$_{d}$ & Tasa periódica de devaluación \\ \hline
i$_{f}$ & Tasa periódica de la inflación \\ \hline
i$_{r}$ & Tasa periódica real \\ \hline

m$_{1}$ & Número de períodos anuales para  i$_{1}$\\ \hline

m$_{2}$ & Número de períodos anuales para  i$_{2}$\\ \hline

n  & Número de periodos asociados a una i\\ \hline

na(120d)v & Nominal anual (120 días) vencido  \\ \hline

naaa & Nominal anual (año) anticipado\\ \hline  

naav & Nominal anual (año) vencido \\ \hline 

nasa & Nominal anual (semestre) anticipado\\\hline 

nasv & Nominal anual (semestre) vencido\\\hline  

nata & Nominal anual (trimestre) anticipado \\ \hline 

natv & Nominal anual (trimestre) vencido \\ \hline    

na(5a)a & Nominal anual (5 años) anticipado \\ \hline

na(5a)v & Nominal anual (5 años) vencido \\ \hline

na(120d)a & Nominal anual (120 días) anticipado  \\ \hline 

p(120d)v & Periódico (120 días) vencido\\ \hline

paa & Periódico (año)  anticipado \\ \hline

pav & Periódico (año) vencido  \\ \hline 

psa & Periódico (semestre) anticipado \\\hline 

psv & Periódico (semestre) vencido \\\hline

pta & Periódico (trimestre) anticipado \\ \hline 

ptv & Periódico (trimestre) vencido \\ \hline      

p(5a)a & Periódico (5 años) anticipado \\ \hline 

p(5a)v & Periódico (5 años) vencido \\ \hline 

p(120d)a & Periódico (120 días) anticipado  \\ \hline  

R & Cuotas seriales fijas  \\ \hline

SPREAD & Tasa de interés que se adiciona a una tasa de referencia \\ \hline




 
\end{tabular}
\end{center}
\end{spacing}
\ \ \
%\begin{center}
%\section*{Aceptaciones Bancarias}
%\addcontentsline{toc}{section}{Tabla resumen de fórmulas }
%\end{center}
\\ \\ \\ \\ \\ \\ \\
\begin{center} \textbf {Aceptaciones Bancarias (capítulo 3)}\\ \end{center} 

\begin{spacing}{1}
\begin{center}
\begin{tabular}{ |p{2.5cm}|p{9.5cm}|}
\hline 
\rowcolor{orange!50}
\begin{center}\textbf{Siglas} \end{center} & \begin{center} \textbf{Nombre} \end{center} \\ \hline     
 

COM$_{v}$ & Comisión de Venta del Corredor\\ \hline 
i$_{c}$ & Tasa de compra\\ \hline 
i$_{R}$ & Tasa de registro en Bolsa de Valores\\ \hline
i$_{v}$ & Tasa de venta\\ \hline  
P$_{c}$ & Precio del comprador\\ \hline  
P$_{R}$ & Precio de registro\\ \hline  
P$_{v}$ & Precio de venta\\ \hline  
$*$ & Para las ecuaciones de valor el * es un operador que multiplica o desplaza un flujo y/o el resultado del flujo de series uniformes y/o gradiantes diferidos y/o con tiempo muerto (único caso de uso de este operador en esta guía.\\ \hline
 

 
\end{tabular}
\end{center}
\end{spacing}

\\ \\ \\

\begin{center} \textbf {Operadores lógicos}\\ \end{center} 

\begin{spacing}{1}
\begin{center}
\begin{tabular}{ |p{2.5cm}|p{9.5cm}|}
\hline 
\rowcolor{orange!50}
\begin{center}\textbf{Siglas} \end{center} & \begin{center} \textbf{Nombre} \end{center} \\ \hline 

$ Q \Leftarrow P $ & "Q dado que P", "Q si P", "Q siempre que P" \\ \hline

$ P \Rightarrow Q $ & "P implica Q", "Si P, entonces Q", "P sólo si Q" \\ \hline

$ P \Leftrightarrow Q $ & "P si y sólo si Q", "P es necesario y suficiente para Q", 
     "P es equivalente con Q"\\ \hline

$ \therefore P $ &  "Porque P" \\ \hline

$ \therefore  P $ & "Por lo tanto, P" \\ \hline

$\equiv$ & Equivalente \\ \hline

\end{tabular}
\end{center}
\end{spacing}
\newpage
 %-----------------------------------------------------------
 %              TABLA DE FORMULAS
 %------------------------------------------------------

\begin{center}
\section*{ii. Tabla resumen de fórmulas}
\addcontentsline{toc}{section}{ii. Tabla resumen de fórmulas trabajadas en la guía}
\end{center}

\begin{spacing}{1}
\begin{center}
\begin{tabular}{ |p{7cm}|p{7cm}|}
\hline 
\rowcolor{orange!50}

\begin{center}\textbf{Fórmula} \end{center} & \begin{center} \textbf{Nombre} \end{center}  \\ \hline

$(1 + i_{1})^{m_1} = (1 + i_{2})^{m_2}$ & Equivalencia de tasas periódicas\\\hline 

$i_{r} = i_{1} + i_{2} + i_{1}i_{2}$ & Tasa de interés periódica real\\ \hline
$i_r = \frac{i - i_f}{1 + i_f}$ & Tasa de interés real\\ \hline 

$j = im$  &  Tasa interés nominal anual vencida\\ \hline   
 

$i_a = \frac{i}{1 + i}$  & Tasa de interés periódica anticipada\\ \hline   

 $i = \frac{i_a}{1 - i_a}$ & Tasa de interés periódica vencida\\ \hline   
$D = Fdn$ & Valor de descuento dado un flujo (F)\\ \hline
  
 $F = P(1 + i)^n$ & Valor futuro\\ \hline 

 $VP = R  \frac{1-(1 + i)^{-n}}{i} $ & Valor presente serie uniforme vencida\\ \hline 

$VP_{a} = R  \frac{1-(1 + i)^{-n}}{i}  (1 + i) $ & Valor presente serie uniforme anticipada\\ \hline 

$VF = R  \frac{(1 + i)^n-1}{i} $ & Valor futuro serie uniforme vencida\\ \hline   
$VF_{a} = R  \frac{(1 + i)^n-1}{i}(1 + i) $ & Valor futuro serie uniforme anticipada\\ \hline 

$VP = \frac{R}{i}$ & Valor presente serie perpetua vencida\\ \hline 
$VP = R  \frac{1-(1 + i)^{-n}}{i} + \frac{L}{i}[ \frac{1-(1 + i)^{-n}}{i} - n(1 + i)^{-n} ]$ & Valor presente de gradiente aritmético\\ \hline 
 $VF = R  \frac{(1 + i)^{n} - 1}{i} + \frac{L}{i}[ \frac{(1 + i)^n - 1}{i} - n ]$ &  Valor futuro de gradiente aritmético\\ \hline 
$VP = \frac{R}{i} + \frac{L}{i^2}$ & Valor presente gradiente aritmético infinito\\ \hline 
$VP = R  \frac{(1 + g)^n  (1 + i)^{-n}-1}{g - i} $ & Valor presente de gradiente geométrico, si $g \neq i$\\ \hline 
$VP = \frac{Rn}{1 + i}$ & Valor presente gradiente geométrico si $g = i$\\ \hline 
$VF = R  \frac{(1 + g)^n - (1 + i)^n}{g - i} $ & Valor futuro gradiente geométrico si $g \neq i$\\ \hline 
$VF = Rn(1 + i)^{n-1} $ & Valor futuro gradiente geométrico si $g = i$\\ \hline 
$VP = \frac{R}{1 - g} $ & Valor presente gradiente geométrico infinito si $g < i$\\ \hline
 $VP = \infty $ & Valor presente gradiente geométrico infinito si $g \geq i$ &\\ \hline 
 
\end{tabular}

\clearpage
		\includegraphics[width = 10.0 cm]{general}\\
		\includegraphics[height=15cm]{TTime}
\end{center}
\end{spacing}

\clearpage

\begin{center}
\section*{iii. Tabla resumen de fórmulas}
\addcontentsline{toc}{section}{iii. Tabla funciones de Excel}
\end{center}

\vspace{2cm}

\begin{spacing}{1}
\begin{center}
\begin{tabular}{ |p{7cm}|p{5.5cm}|}
\hline 
\rowcolor{orange!50}

\begin{center}\textbf{Nombre} \end{center}  & \begin{center} \textbf{Función Financiera en Excel} \end{center}  \\ \hline

Equivalencia de tasas periódicas &  TASA(m;;-1;1+i)\\\hline 

Tasa interés nominal anual vencida & TASA.NOMINAL(i;m) \\ \hline

Tasa interés periódica año vencido & INT.EFECTIVO(j;m) \\ \hline

Valor futuro & VF(i;n;;VA,0)\\ \hline 

Valor presente & VA(i;n;;VF,0)\\ \hline 

Valor presente serie uniforme vencida & VNA(i;R1;R2;R3;...) \\ \hline  

Valor futuro serie uniforme vencida &  VF(i;n;;VA,0)\\ \hline

Valor futuro serie uniforme anticipada & VF(i;n;;VA,1)\\ \hline 

Valor presente serie perpetua vencida & VA(i;n;R;0) \\ \hline

Valor presente gradiente aritmético & VNA(i;R1;R2;R3;...) \\ \hline

Valor presente gradiente geométrico & VNA(i;R1;R2;R3;...) \\ \hline

Valor cuota serie uniforme vencida & PAGO(i;n;P;F;0) \\ \hline

Valor cuota serie uniforme anticipada & PAGO(i;n;P;F;1) \\ \hline

Ecuación de valor & Función Buscar Objetivo\\ \hline


\end{tabular}
\end{center}
\end{spacing}

\clearpage